
\documentclass{article}
%\usepackage{graphicx}
\usepackage{amsmath}
\usepackage{mwe}
\usepackage{subfig}
\usepackage{float}
\usepackage{xcolor}

\usepackage{graphicx,floatpag,fancyhdr}
\usepackage{lipsum}

% This defines the fancy page style to be similar to plain
\pagestyle{fancy}
\fancyhf{}% Clear header/footer
\fancyfoot[C]{\thepage}
\renewcommand{\headrulewidth}{0pt}% Remove header rule

% Page style plainlower is similar to plain, but lowers page number by 6 lines of text
\fancypagestyle{plainlower}{
\fancyhf{}% Clear header/footer
\fancyfoot[C]{\raisebox{-6\baselineskip}{\thepage}}
\renewcommand{\headrulewidth}{0pt}% Remove header rule
}

%All LaTeX documents have a ``preamble'' that includes the packages and macros needed to make the document compile. The file `PomonaLgcsFormatting.tex' includes the preamble for this template. You can see it in the file list on the left frame of your screen, and this document is instructed to use it with the \input{} command below.


\title{Ny-Alesund O3S-DQA Homogenization Report}
\author{Deniz Poyraz}
\date{\today}

\begin{document}

\maketitle

\section{Ny-Alesund Metadata Timeseries}
\label{sec:metadata}
Ny-Alesund ozonesonde time series starts in 1992-01. Ozonesonde data were downloaded from the NDACC server and
DQA homogenization is processed by RMI. Relevant metadata values are extracted from the header of ames data files
and from the files provided by the station PI. Background (iB2) and pump flow (PF) rate values are missing for 1997.
For missing values, the median of the relevant value is used.
The temperature (TLab), humidity (RHLab) and pressure (PLab) values of the laboratory are used for humidity correction.
When one of these measurements is missing, climatological averages are calculated for each month and these values
are used for the missing relevant metadata.
TLab, PLab and ULab are used for pump flow rate moisture correction. iB2 is used for background correction.
Related plots are shown in \ref{fig:iB2} - \ref{fig:PLab}.




\begin{figure}
\centering
\includegraphics[width=0.9\textwidth]{../../Files/ny-aalesund/Plots/Metadata/iB2}
\caption{Ny-Alesund iB2 timeseries}
\label{fig:iB2}
\end{figure}

\begin{figure}
\centering
\includegraphics[width=0.9\textwidth]{../../Files/ny-aalesund/Plots/Metadata/PF}
\caption{Ny-Alesund pump flow rate timeseries}
\label{fig:PF}
\end{figure}



\begin{figure}
\centering
\includegraphics[width=0.9\textwidth]{../../Files/ny-aalesund/Plots/Metadata/TLab}
\caption{Ny-Alesund laboratory temperature timeseries}
\label{fig:TLab}
\end{figure}

\begin{figure}
\centering
\includegraphics[width=0.9\textwidth]{../../Files/ny-aalesund/Plots/Metadata/RHLab}
\caption{Ny-Alesund laboratory humidity timeseries}
\label{fig:ULab}
\end{figure}

\begin{figure}
\centering
\includegraphics[width=0.9\textwidth]{../../Files/ny-aalesund/Plots/Metadata/PLab}
\caption{Ny-Alesund laboratory humidity timeseries}
\label{fig:PLab}
\end{figure}

\section{O3S Corrections}
\label{sec:v04}


The recommended and applied O3S-DQA corrections are summarized below.
\begin{enumerate}
\item Conversion efficiency
\item Background current
\item Pump temperature measurement
\item Pump flow rate, moistening effect
\item Pump flow efficiency at low pressures
\item Total ozone normalization: in O3S-DQA guide this correction factor is recommended to be added in the data-set,
but the normalization factor is applied.
\item Radiosonde changes: RS80 radiosonde correction is tested but not applied yet.
\end{enumerate}


O3S-DQA corrections are applied to the raw current measured by ECC's. The raw current values are
determined from converting partial ozone values available
in the NDACC files to ECC current for the pre-2017 and are available in ames files starting from 2017
and onwards.

The pump temperature values, pump flow rate and background values (iB2) are needed
to obtain the raw current values.
The corrections applied to calculate ozone partial pressure values in NDACC files, are un-corrected to
get the raw ECC current values. These corrections, applied in the Vaisala software, are:
pressure dependent background correction for SPC sondes and pump flow
efficiency correction.The pump flow efficiency correction table used at this stage is slightly different than
the table used for O3S-DQA pump efficiency corrections.
The correction applied for uncorrecting NDACC pump flow efficiency can be seen in Ozone Sounding with Vaisala
Radiosonde RS41 User's Guide M211486EN, page 74 and the correction table
used for O3S-DQA can be seen in O3S-DQA Activity: Guide Lines for Homogenization of Ozone Sonde Data (Version 2.0)
at page 34.
The Ozone partial values values from the PI station are shown as
'NDACC' and all DQA corrections are denoted by 'DQA'
in the rest of the report.

%
\subsubsection{Conversion efficiency}
No absorption efficiency and stoichiometry correction are needed since 3ml of cathode solution and SPC
$1.0\%-1.0$B probes are used throughout the time series.
%        efficiency correction is applied.
%
%%

\subsubsection{Background current}
The effect of background correction on Ny-Alesund data, using iB2, is shown in Figure~\ref{fig:bkg}.
    If $I_B$ exceeds $I_{B,\text{Mean}}+2\sigma_{IB}$, then $I_B$
    should be replaced by a more representative climatic value $I_{B,\text{Mean}}$, but with a
    larger uncertainty $2\sigma_{IB}$, Figure~\ref{fig:iB2}.


\begin{figure}
\centering
\includegraphics[width=1.3\textwidth]{../../Files/ny-aalesund/DQA_nors80/Binned/Plots/EtaBkg_vs_Eta_alltimerange.png}
\caption{Background current correction}
\label{fig:bkg}
\end{figure}
%%
\subsubsection{Pump temperature measurement}

The truest pump temperature correction is applied according to Eq.13 in the O3S-DQA Guidelines.
Until 1996-12 SPC-5A sondes, and onwards  SPC-6A sondes have been launched.
These periods need different corrections and the effects are shown in Figure~\ref{fig:tpump}.


\begin{figure}
\centering
\includegraphics[width=1.2\textwidth]{../../Files/ny-aalesund/DQA_nors80/Binned/Plots/EtaBkgTpump_vs_EtaBkg_alltimerange.png}
\caption{Pump temperature correction }
\label{fig:tpump}
\end{figure}
%%
\subsubsection{Pump flow rate (moistening effect)}
This correction, Eq.15 of the O3S-DQA Guidelines, is applied and shown in Fig~\ref{fig:pf_ptu}.
Details of the values used for this correction are described in Sec~\ref{sec:metadata}.

%
\begin{figure}
\centering
\includegraphics[width=1.2\textwidth]{../../Files/ny-aalesund/DQA_nors80/Binned/Plots/EtaBkgTpumpPhigr_vs_EtaBkgTpump_alltimerange.png}
\caption{Pump flow rate correction applied}
\label{fig:pf_ptu}
\end{figure}
%%
\subsubsection{Pump flow efficiency}
This correction, Eq.22 of the O3S-DQA Guidelines, is applied using Table 6 of the O3S-DQA Guidelines and
shown in Fig~\ref{fig:pf_eff}.
%The interpolation of the correction factors are made using the pressure. This method gives the same result as doing the interpolation using the logarithm of pressure
%and polynomial fit with an error of less than $0.03\%$.
The effect of this correction is shown in Fig~\ref{fig:pf_eff}.
%
\begin{figure}
\centering
\includegraphics[width=1.2\textwidth]{../../Files/ny-aalesund/DQA_nors80/Binned/Plots/EtaBkgTpumpPhigrPhiEff_vs_EtaBkgTpumpPhigr_alltimerange.png}
\caption{Pump flow rate correction applied}
\label{fig:pf_eff}
\end{figure}
%%

%
The effect of all DQA corrections with respect to no correction (Raw PO3) is shown in Fig~\ref{fig:dqa_all}
and comparison of DQA corrected and
NDACC O3 values is shown in Fig~\ref{fig:fig_dqa_ndacc}.

\begin{figure}
\centering
\includegraphics[width=1.2\textwidth]{../../Files/ny-aalesund/DQA_nors80/Binned/Plots/DQA_vs_Raw_alltimerange.png}
\caption{Effect of all DQA corrections}
\label{fig:dqa_all}
\end{figure}
%
\begin{figure}
\centering
\includegraphics[width=1.2\textwidth]{../../Files/ny-aalesund/DQA_nors80/Binned/Plots/DQA_vs_NDACC_alltimerange.png}
\caption{Comparison of DQA and NDACC O3S values}
\label{fig:fig_dqa_ndacc}
\end{figure}
%%%%
%%%
 \subsubsection{Radiosonde correction}
RS80 radiosonde correction is applied to correct the pressure offset difference.
This correction is not included in the DQA corrections, it is only applied to see its effect.
%It's uncertainity is not implemented in the
%total uncertainity of the ozone partial pressure and also in the final DQA ver.
The effect of RS80 correction is shown in Fig~\ref{fig:rs80}. \\

                        \begin{figure}
        \centering
\includegraphics[width=1.2\textwidth]{../../Files/ny-aalesund/DQA_nors80/Binned/Plots/RS80_vs_noRS80.png}
    \caption{RS80 correction applied}
            \label{fig:rs80}
    \end{figure}
%
%
\section{Comparison plots to AURA MLS v04}

The homogenized and non-homogenized Ny-Alesund data is compared with AURA-MLS data using v04. The
NDACC, DQA-homogenized O3 data sets are compared and shown in figures between
\ref{fig:niwav04} and \ref{fig:dqav04_rs80}.

\begin{figure}
\centering
\includegraphics[width=1.2\textwidth]{../../Files/ny-aalesund/MLS/Plots/NDACC_vs_MLS_v04_nors80.png}
\caption{ NDACC Ny-Alesund O3S vs AURA MLS v04  }
\label{fig:niwav04}
\end{figure}


\begin{figure}
\centering
\includegraphics[width=1.2\textwidth]{../../Files/ny-aalesund/MLS/Plots/DQA_vs_MLS_v04_nors80.png}
\caption{DQA Ny-Alesund O3S vs AURA MLS v04 }
\label{fig:dqav04_rs80}
\end{figure}

        \section{Total Ozone and Total Ozone Normalization Factors}

Total Ozone Normalization (TON) factors have been calculated with and without DQA corrections. TON factors are
extracted
from taking the ratio of TO values from sonde to the TO values given by satellites. TO from the sonde is integrated until
burst pressure (max. 10hPa) and the residuals, calculated from climatological means, are added.
The corresponding plots are shown in Fig.\ref{fig:ton1} and  Fig.\ref{fig:ton2}.

                                \begin{figure}
        \centering
\includegraphics[width=1.2\textwidth]{../../Files/ny-aalesund/Plots/TON/TON_sonde_nors80.png}
    \caption{TO values calculated with NDACC and DQA corrected ozonesonde data}
            \label{fig:ton1}
    \end{figure}

                                 \begin{figure}
        \centering
\includegraphics[width=1.2\textwidth]{../../Files/ny-aalesund/Plots/TON/TON_satellite_one.png}
    \caption{Relative differences of TON values with respect to satellite data calculated with NDACC and DQA corrected ozonesonde data}
            \label{fig:ton2}
    \end{figure}



\end{document}
