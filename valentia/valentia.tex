
\documentclass{article}
%\usepackage{graphicx}
\usepackage{amsmath}
\usepackage{mwe}
\usepackage{subfig}
\usepackage{float}
\usepackage{xcolor}

\usepackage{graphicx,floatpag,fancyhdr}
\usepackage{lipsum}
\usepackage{amssymb}

% This defines the fancy page style to be similar to plain
\pagestyle{fancy}
\fancyhf{}% Clear header/footer
\fancyfoot[C]{\thepage}
\renewcommand{\headrulewidth}{0pt}% Remove header rule

% Page style plainlower is similar to plain, but lowers page number by 6 lines of text
\fancypagestyle{plainlower}{
  \fancyhf{}% Clear header/footer
  \fancyfoot[C]{\raisebox{-6\baselineskip}{\thepage}}
  \renewcommand{\headrulewidth}{0pt}% Remove header rule
}

%All LaTeX documents have a ``preamble'' that includes the packages and macros needed to make the document compile. The file `PomonaLgcsFormatting.tex' includes the preamble for this template. You can see it in the file list on the left frame of your screen, and this document is instructed to use it with the \input{} command below.


\title{Valentia O3S-DQA Homogenization Report}
\author{Deniz Poyraz}
\date{\today}

\begin{document}

\maketitle

\section{Valentia Metadata Timeseries}
\label{sec:metadata}

Valentia data is dowloaded from WOUDC data server and processed by RMI. The timeseries starts at 1994/01.
The temperature (TLab), humidity (ULab) and pressure (PLab) of the laboratory measurements are missing before 2003.
For the missing values, monthly climatological means are calculated with the available data and these values are used for the corresponding missing metadata.
TLab, PLab and ULab are used for pump flow rate humidity correction. For the background correction, iB2 is used.
The related plots are shown in Figs \ref{fig:iB1} - \ref{fig:PLab}.



    \begin{figure}
        \centering
\includegraphics[width=0.9\textwidth]{../../Files/valentia/Plots/Metadata/iB1}
%        \includegraphics[width=0.9\textwidth]{../../Files/valentia/MLS/Plots/Raw_vs_MLS_zoom.pdf}
    \caption{Valentia iB1 timeseries}
            \label{fig:iB1}
    \end{figure}

    \begin{figure}
        \centering
\includegraphics[width=0.9\textwidth]{../../Files/valentia/Plots/Metadata/iB2}
%        \includegraphics[width=0.9\textwidth]{../../Files/valentia/MLS/Plots/Raw_vs_MLS_zoom.pdf}
    \caption{Valentia iB2 timeseries}
            \label{fig:iB2}
    \end{figure}

    \begin{figure}
        \centering
\includegraphics[width=0.9\textwidth]{../../Files/valentia/Plots/Metadata/PF}
%        \includegraphics[width=0.9\textwidth]{../../Files/valentia/MLS/Plots/Raw_vs_MLS_zoom.pdf}
    \caption{Valentia pump flow rate timeseries}
            \label{fig:PF}
    \end{figure}



           \begin{figure}
        \centering
\includegraphics[width=0.9\textwidth]{../../Files/valentia/Plots/Metadata/TLab}
%        \includegraphics[width=0.9\textwidth]{../../Files/valentia/MLS/Plots/Raw_vs_MLS_zoom.pdf}
    \caption{Valentia laboratory temperature timeseries}
            \label{fig:TLab}
    \end{figure}

               \begin{figure}
        \centering
\includegraphics[width=0.9\textwidth]{../../Files/valentia/Plots/Metadata/ULab}
%        \includegraphics[width=0.9\textwidth]{../../Files/valentia/MLS/Plots/Raw_vs_MLS_zoom.pdf}
    \caption{Valentia laboratory humidity timeseries}
            \label{fig:ULab}
    \end{figure}

               \begin{figure}
        \centering
\includegraphics[width=0.9\textwidth]{../../Files/valentia/Plots/Metadata/Pground}
%        \includegraphics[width=0.9\textwidth]{../../Files/valentia/MLS/Plots/Raw_vs_MLS_zoom.pdf}
    \caption{Valentia laboratory humidity timeseries}
            \label{fig:PLab}
    \end{figure}

\section{O3S Corrections}
\label{sec:v05}

\textcolor{blue}{Assumptions made for homogenization are indicated in blue, to be confirmed by the station PI}\\
\textcolor{red}{Unusual behaviours are indicated in red}
%
%
The recommended and applied O3S-DQA corrections are summarized below.\\
    \begin{enumerate}
        \item Conversion efficiency \\
         (\textcolor{blue}{Transfer functions are not applied since always SPC 1.0$\%$-1.0B sondes have been used.
         3ml of cathode solution is used during the entire period that the absorption efficiency is taken as 1.
 })
        \item Background current\\
        (\textcolor{blue}{iB2 is used through entire time-series})\\
%        (\textcolor{red}{iB2 values are very variable, especially for the period before 1995 and after 2017})\\
        \item Pump temperature measurement\\
        (\textcolor{blue}{For 5A SPC sondes, pump temperature thermistor was taped externally and for 6A and ENSCI sondes
        was taped internally. })\\
        \item Pump flow rate, moistening effect\\
                (\textcolor{blue}{Humidty correction is not applied in the Vaisala software that the WOUDC O3 values are
                not corrected for humidity.})\\
%        (\textcolor{red}{There is a large variety in PF rate values for the period before 2003})\\

        \item Pump flow efficiency at low pressures\\
        (\textcolor{blue}{PF efficiency tables are applied as in the Vaisala software.
        These are:\\
SPC Table = [ 1.171, 1.171, 1.131, 1.092, 1.055, 1.032, 1.022, 1.015, 1.011, 1.008, 1.006, 1.004,    1,    1]\\
         for Pressure  =    [    0,     2,     3,      5,    10,    20,    30,    50,   100,   200,   300,   500, 1000, 1100] values and
        ENSCI Table = [1.24, 1.24, 1.124, 1.087, 1.066, 1.048, 1.041, 1.029, 1.018, 1.013,  1.007, 1.002,     1,    1]\\
        for Pressure = [   0,    3,     5,     7,    10,    15,    20,    30,    50,    70,    100,   150,   200, 1100] values.
 })\\
        \item Total ozone normalization: in O3S-DQA guide this correction factor is recommended to be added in the data-set,
        but the normalization factor is not applied.\\
          \subitem{$\bullet $ TO values from Dobson/Brewer spectrophotometers are taken from the WOUDC data-server.}\\
%        \item Radiosonde changes: RS80 radiosonde correction is tested but not applied in the DQA homogenization.\\
%          (\textcolor{blue}{RS80 radiosondes were used from 1989-02 till 2007-01, RS92 used from 2007-01 till 2015-12 and RSA411
%          are used afterwards.})\\
\end{enumerate}

The O3S-DQA corrections are applied to the raw current measured by the ECC's. The raw current values are
determined from converting partial ozone values that are
in the WOUDC files to current. Knowing pump temperature values, pump flow rate and background values (iB2) is essential
to obtain the raw current values.
The corrections that were applied to have the WOUDC ozone partial pressure values, are un-corrected to
get the raw current values. These un-corrections correspond to the corrections applied in the Vaisala software, which are:
pressure dependent background correction for SPC sondes and pump flow
efficiency correction. The pump flow efficiency correction table used at this stage is slightly different than the table used
for O3S-DQA pump efficiency corrections.
The correction applied for uncorrecting the WOUDC pump flow efficiency can be seen in Ozone Sounding with Vaisala
Radiosonde RS41 User's Guide M211486EN, page 74 and the correction table
used for O3S-DQA can be seen in O3S-DQA Activity: Guide Lines for Homogenization of Ozone Sonde Data (Version 2.0)
at page 34.

The ozone partial pressure values converted from raw current
wihout applying any correction are denoted as 'Raw' or 'No correction', the O3 values from WOUDC files are denoted
by 'WOUDC' and the ozone partial values that have all the DQA corrections
are denoted by 'DQA' in the rest of the manuscript.

%
    \subsubsection{Conversion efficiency}
3ml of cathode solution is used all entire period that the absorption efficiency is taken as 1.
Since SPC 1.0$\%$-1.0B sondes have been used, stoichmetry correction is not needed. Therefore,
the conversion efficiency is taken 1 and this correction do not have an effect on the data-set, Fig~\ref{fig:eta}

%
%%
                \begin{figure}
        \centering
\includegraphics[width=1.3\textwidth]{../../Files/valentia/DQA_nors80/Binned/Plots/Eta_vs_Raw_alltimerange.png}
    \caption{Conversion efficiency correction}
            \label{fig:eta}
    \end{figure}

%%%
%%%
        \subsubsection{Background current}
        Background correction, iB2, applied to Valentia time-series is shown in Fig~\ref{fig:iB2}. If $I_B$ exceeds
$I_{B,\text{Mean}}+2\sigma_{IB}$ then $I_B$, it
should be replaced by the more representative climatological value of $I_{B,\text{Mean}}$, however with
larger uncertainty of $2\sigma_{IB}$.
Therefore if $I_B$ values are falling above $I_{B,\text{Mean}}+2\sigma_{IB}$ in Fig~\ref{fig:iB2}, the background correction is applied.
%For the mean and standard deviations
%of the iB2 values, 3 different periods are considered.
%As it can be seen in Fig~\ref{fig:iB2}, iB2 values are larger for the period between 1993 and 1996 and after 2017.
%Therefore the mean and corresponding standard deviations are calculated and applied separately in these periods.
%%
%%
                \begin{figure}
        \centering
\includegraphics[width=1.3\textwidth]{../../Files/valentia/DQA_nors80/Binned/Plots/EtaBkg_vs_Eta_alltimerange.png}
    \caption{Background current correction}
            \label{fig:bkg}
    \end{figure}
%%
            \subsubsection{Pump temperature measurement}
 Truest pump temperature correction is applied according to Eq.13 of the O3S-DQA Guidelines. Until 2003 SPC-5A sondes,
 and SPC 6A sondes have been launched. After 2003, 6A sondes have been used. 5A and 6A sonde periods need different corrections and
 their effects are shown in Fig~\ref{fig:tpump} as dark and light red respectively.

% At 1998-12-02 the pump location
% changed from being in the box to the inside the pump.
%    Therefore Case-III correction is applied to SPC-5A sondes and case-V correction to SPC-6A sondes. The effect of the temperature correction
%is shown in  Fig~\ref{fig:tpump}.
%%
%
                    \begin{figure}
        \centering
\includegraphics[width=1.2\textwidth]{../../Files/valentia/DQA_nors80/Binned/Plots/EtaBkgTpump_vs_EtaBkg_alltimerange.png}
    \caption{Pump temperature correction }
            \label{fig:tpump}
    \end{figure}
%%
                \subsubsection{Pump flow rate (moistening effect)}
    This correction, Eq.15 of the O3S-DQA Guidelines, is applied and shown in Fig~\ref{fig:pf_ptu}. The details of the
values used for correction is explained in Sec~\ref{sec:metadata}. Considering that the humidity values of the laboratorty
is very high, the effect of this correction is not very significant.
%
                        \begin{figure}
        \centering
\includegraphics[width=1.2\textwidth]{../../Files/valentia/DQA_nors80/Binned/Plots/EtaBkgTpumpPhigr_vs_EtaBkgTpump_alltimerange.png}
    \caption{Pump flow rate humidity correction applied}
            \label{fig:pf_ptu}
    \end{figure}
%%
                   \subsubsection{Pump flow efficiency}
    This correction, Eq.22 of the O3S-DQA Guidelines, is applied using Table 6 of the O3S-DQA Guidelines and
    shown in Fig~\ref{fig:pf_eff}.
%The interpolation of the correction factors are made using the pressure. This method gives the same result as doing the interpolation using the logarithm of pressure
%and polynomial fit with an error of less than $0.03\%$.
The effect of this correction is shown in Fig~\ref{fig:pf_eff}.
%
                        \begin{figure}
        \centering
\includegraphics[width=1.2\textwidth]{../../Files/valentia/DQA_nors80/Binned/Plots/EtaBkgTpumpPhigrPhiEff_vs_EtaBkgTpumpPhigr_alltimerange.png}
    \caption{Pump flow rate correction applied}
            \label{fig:pf_eff}
    \end{figure}
%%

%%
The effect of all DQA correction with respect to no correction is shown in Fig~\ref{fig:dqa_all} and the comparison of DQA corrected and
WOUDC O3S values is shown in Fig~\ref{fig:fig_dqa_ndacc}.

                        \begin{figure}
        \centering
\includegraphics[width=1.2\textwidth]{../../Files/valentia/DQA_nors80/Binned/Plots/DQA_vs_Raw_alltimerange.png}
%        \includegraphics[width=0.9\textwidth]{../../Files/valentia/MLS/Plots/Raw_vs_MLS_zoom.pdf}
    \caption{Effect of all DQA corrections}
            \label{fig:dqa_all}
    \end{figure}
%
                        \begin{figure}
        \centering
\includegraphics[width=1.2\textwidth]{../../Files/valentia/DQA_nors80/Binned/Plots/DQA_vs_NDACC_alltimerange.png}
%        \includegraphics[width=0.9\textwidth]{../../Files/valentia/MLS/Plots/Raw_vs_MLS_zoom.pdf}
    \caption{Comparison of DQA and WOUDC O3S values}
            \label{fig:fig_dqa_ndacc}
    \end{figure}
%%%
%%%
% \subsubsection{Radiosonde correction}
%    The RS80 radiosonde correction is applied to correct the pressure offset difference.
%    This correction is not included in the DQA corrections, but only applied to see its effect.
%%It's uncertainity is not implemented in the
%%total uncertainity of the ozone partial pressure and also in the final DQA ver.
%    The effect of RS80 correction is shown in Fig~\ref{fig:rs80}. \\
%
%                        \begin{figure}
%        \centering
%\includegraphics[width=1.2\textwidth]{../../Files/valentia/DQA_nors80/Binned/Plots/RS80_vs_noRS80.png}
%%        \includegraphics[width=0.9\textwidth]{../../Files/valentia/MLS/Plots/Raw_vs_MLS_zoom.pdf}
%    \caption{RS80 correction applied}
%            \label{fig:rs80}
%    \end{figure}
%%
    \section{Effect DQA corrections on vertical ozone profiles }
%
    In order to see the effect of DQA corrections, explained in Sec.\ref{sec:v05}, on the ozone profiles
the ozone profiles have been
explored with and without DQA corrections. In Fig\ref{fig:pl_ndacc}, the ozone profiles with DQA and
WOUDC corrections are shown.
%%
                                \begin{figure}
        \centering
\includegraphics[width=1.2\textwidth]{../../Files/valentia/Plots/hom_vs_woudc.png}
%        \includegraphics[width=0.9\textwidth]{../../Files/valentia/MLS/Plots/Raw_vs_MLS_zoom.pdf}
    \caption{DQA corrections and WOUDC corrections}
            \label{fig:pl_ndacc}
    \end{figure}

\section{Comparison plots to AURA MLS v05}

    The homogenized and non-homogenized Valentia data is compared with AURA-MLS data using v05. The
    WOUDC and DQA homogenized O3 data sets are compared and shown in figures between
\ref{fig:niwav05} and \ref{fig:dqav05}.

                                \begin{figure}
        \centering
\includegraphics[width=1.2\textwidth]{../../Files/valentia/MLS/Plots/Original_vs_MLS_v05.png}
%        \includegraphics[width=0.9\textwidth]{../../Files/valentia/MLS/Plots/Raw_vs_MLS_zoom.pdf}
    \caption{ WOUDC Valentia O3S vs AURA MLS v05  }
            \label{fig:niwav05}
    \end{figure}

                            \begin{figure}
        \centering
\includegraphics[width=1.2\textwidth]{../../Files/valentia/MLS/Plots/DQA_vs_MLS_v05.png}
%        \includegraphics[width=0.9\textwidth]{../../Files/valentia/MLS/Plots/Raw_vs_MLS_zoom.pdf}
    \caption{DQA Valentia O3S vs AURA MLS v05 }
            \label{fig:dqav05}
    \end{figure}

%
        \section{Total Ozone and Total Ozone Normalization Factors}

The Total Ozone Normalization (TON) factors have been calculated with and without DQA corrections. For the TON
the ratio of TO from the sonde to the TO from satellites are taken. For the TO from the sonde, the TO is integrated until
burst (max. 10hPa) and the residuals, calculated from climatological means, are added.
The corresponding plots are shown in Fig.\ref{fig:ton1} and  Fig.\ref{fig:ton3}.

                                \begin{figure}
        \centering
\includegraphics[width=1.2\textwidth]{../../Files/valentia/Plots/TON/TON_RDif_sonde_nors80.png}
    \caption{TO values calculated with WOUDC and DQA corrected ozonesonde data}
            \label{fig:ton1}
    \end{figure}
%
                                 \begin{figure}
        \centering
\includegraphics[width=1.2\textwidth]{../../Files/valentia/Plots/TON/TON_Brewer_comparison.png}
    \caption{TO from Brewer and Relative differences of TO values with respect to Brewer calculated with WOUDC
    and DQA corrected ozonesonde data}
            \label{fig:ton2}
    \end{figure}

                                     \begin{figure}
        \centering
\includegraphics[width=1.2\textwidth]{../../Files/valentia/Plots/TON/TON_satellite_one.png}
    \caption{Relative differences of TON values with respect to satellite data calculated with WOUDC and DQA corrected ozonesonde data}
            \label{fig:ton3}
    \end{figure}

            \section{Pump Temperature Profiles}
    The pump temperature values are investigated to check if there are any anamolies during the entire time series.
    The related plots are shown in Fig.\ref{fig:av_tp}-\ref{fig:tp_30}.

                                         \begin{figure}
        \centering
\includegraphics[width=1.2\textwidth]{../../Files/valentia/Plots/TPump/Averaged_TPump.png}
    \caption{Pump temperature profiles as a function of altitude with and without DQA corrections}
            \label{fig:av_tp}
    \end{figure}

%
                                             \begin{figure}
        \centering
\includegraphics[width=1.2\textwidth]{../../Files/valentia/Plots/TPump/All_v2_10.png}
    \caption{Pump temperature values at 10km with and without DQA corrections}
            \label{fig:tp_10}
    \end{figure}

                                                 \begin{figure}
        \centering
\includegraphics[width=1.2\textwidth]{../../Files/valentia/Plots/TPump/All_v2_20.png}
    \caption{Pump temperature values at 10km with and without DQA corrections}
            \label{fig:tp_20}
    \end{figure}

                                                 \begin{figure}
        \centering
\includegraphics[width=1.2\textwidth]{../../Files/valentia/Plots/TPump/All_v2_30.png}
    \caption{Pump temperature values at 10km with and without DQA corrections}
            \label{fig:tp_30}
    \end{figure}

\end{document}
